\documentclass[a4paper]{article}

\usepackage{inputenc}
\usepackage[british,UKenglish]{babel}
\usepackage{amsmath}
%\usepackage{titlesec}
\usepackage{color}
\usepackage{graphicx}
\usepackage{fancyref}
\usepackage{hyperref}
\usepackage{float}
\usepackage{scrextend}
\usepackage{setspace}
\usepackage{xargs}
\usepackage{multicol}
\usepackage{nameref}

\usepackage{sectsty}
\usepackage{multicol}
\usepackage{multirow}
\usepackage[procnames]{listings}
\usepackage{appendix}

\newcommand\tab[1][1cm]{\hspace*{#1}}
\hypersetup{colorlinks=true, linkcolor=black}
\interfootnotelinepenalty=10000

\newcommand{\cleancode}[1]{\begin{addmargin}[3em]{3em}\texttt{\textcolor{cleanOrange}{#1}}\end{addmargin}}
\newcommand{\cleanstyle}[1]{\text{\textcolor{cleanOrange}{\texttt{#1}}}}


\usepackage[colorinlistoftodos,prependcaption,textsize=footnotesize]{todonotes}
\newcommandx{\commred}[2][1=]{\textcolor{Red}
{\todo[linecolor=red,backgroundcolor=red!25,bordercolor=red,#1]{#2}}}
\newcommandx{\commblue}[2][1=]{\textcolor{Blue}
{\todo[linecolor=blue,backgroundcolor=blue!25,bordercolor=blue,#1]{#2}}}
\newcommandx{\commgreen}[2][1=]{\textcolor{OliveGreen}{\todo[linecolor=OliveGreen,backgroundcolor=OliveGreen!25,bordercolor=OliveGreen,#1]{#2}}}
\newcommandx{\commpurp}[2][1=]{\textcolor{Plum}{\todo[linecolor=Plum,backgroundcolor=Plum!25,bordercolor=Plum,#1]{#2}}}

\def\code#1{{\tt #1}}

\def\note#1{\noindent{\bf [Note: #1]}}

\makeatletter
%% The "\@seccntformat" command is an auxiliary command
%% (see pp. 26f. of 'The LaTeX Companion,' 2nd. ed.)
\def\@seccntformat#1{\@ifundefined{#1@cntformat}%
   {\csname the#1\endcsname\quad}  % default
   {\csname #1@cntformat\endcsname}% enable individual control
}
\let\oldappendix\appendix %% save current definition of \appendix
\renewcommand\appendix{%
    \oldappendix
    \newcommand{\section@cntformat}{\appendixname~\thesection\quad}
}
\makeatother


% "define" Scala
\usepackage[T1]{fontenc}  
\usepackage[scaled=0.82]{beramono}  
\usepackage{microtype} 

\sbox0{\small\ttfamily A}
\edef\mybasewidth{\the\wd0 }

\lstdefinelanguage{scala}{
  morekeywords={abstract,case,catch,class,def,%
    do,else,extends,false,final,finally,%
    for,if,implicit,import,match,mixin,%
    new,null,object,override,package,%
    private,protected,requires,return,sealed,%
    super,this,throw,trait,true,try,%
    type,val,var,while,with,yield},
  sensitive=true,
  morecomment=[l]{//},
  morecomment=[n]{/*}{*/},
  morestring=[b]",
  morestring=[b]',
  morestring=[b]"""
}

\usepackage{color}
\definecolor{dkgreen}{rgb}{0,0.6,0}
\definecolor{gray}{rgb}{0.5,0.5,0.5}
\definecolor{mauve}{rgb}{0.58,0,0.82}

% Default settings for code listings
\lstset{frame=tb,
  language=scala,
  aboveskip=3mm,
  belowskip=3mm,
  showstringspaces=false,
  columns=fixed, % basewidth=\mybasewidth,
  basicstyle={\small\ttfamily},
  numbers=none,
  numberstyle=\footnotesize\color{gray},
  % identifierstyle=\color{red},
  keywordstyle=\color{blue},
  commentstyle=\color{dkgreen},
  stringstyle=\color{mauve},
  frame=single,
  breaklines=true,
  breakatwhitespace=true,
  procnamekeys={def, val, var, class, trait, object, extends},
  procnamestyle=\ttfamily\color{red},
  tabsize=2
}

\lstnewenvironment{scala}[1][]
{\lstset{language=scala,#1}}
{}
\lstnewenvironment{cpp}[1][]
{\lstset{language=C++,#1}}
{}
\lstnewenvironment{bash}[1][]
{\lstset{language=bash,#1}}
{}
\lstnewenvironment{verilog}[1][]
{\lstset{language=verilog,#1}}
{}



\lstset{frame=,basicstyle={\footnotesize\ttfamily}}



\graphicspath{ {images/} }
\usepackage{ctex}
\usepackage{verbatim}
\usepackage{geometry}
\usepackage{amsmath}
\usepackage{pifont}%\ding{192} \ding{172}
\usepackage{tikz}
\usepackage{float}
\usepackage{booktabs}
\usepackage{bm}
\usepackage{siunitx}
\usepackage{enumerate}
\usepackage{cite}
%\usepackage{graphicx}
%\geometry{a4paper, scale=0.72}
\geometry{a4paper,left=2.5cm,right=2.5cm,top=2.5cm,bottom=2.5cm}
%%%%%%%%%%%%%%%%%%%%%%%%%%%%%%%%%%%%%%%% BEGIN DOC %%%%%%%%%%%%%%%%%%%%%%%%%%%%%%%%%%%%%%%%

\begin{document}
\renewcommand{\contentsname}{目\ 录}
\renewcommand{\appendixname}{附录}
\renewcommand{\appendixpagename}{附录}
\renewcommand{\refname}{参考文献} 
\renewcommand{\figurename}{图}
\renewcommand{\tablename}{表}
\renewcommand{\today}{\number\year 年 \number\month 月 \number\day 日}
\newcommand{\refeq}[1]{\textbf{Eq.(\ref{#1})}}
\newcommand*{\circled}[1]{\lower.7ex\hbox{\tikz\draw (0pt, 0pt)%
    circle (.5em) node {\makebox[1em][c]{\small #1}};}}
    
\title{{\Huge 近代物理实验报告{\large\linebreak\\}}{\Large 实验10-1:\ 非线性热对流斑图\linebreak\linebreak}}
%please write your name, Student #, and Class # in Authors, student ID, and class # respectively
\author{\\姓\ 名:付\ 大\ 为\\
学\ 号: 1800011105\\
邮\ 箱: \url{fudw@pku.edu.cn}\\
%班\ 号: xxxxx\\\\
近代物理实验 (II)\\
(2022,春季学期)\\\\
北京大学\\
物理学院\\
2018级1班}
\date{\today}
\maketitle
\newpage

%%%%%%%%%%%%%%%%%%%%%%%%%%%%%%%%%%%%%%%% ABSTRACT %%%%%%%%%%%%%%%%%%%%%%%%%%%%%%%%%%%%%%%%
\begin{center}
{\Large\bf{摘\ 要\\}}
\end{center}

热对流是自然界中常见的现象,具有通常所熟知的“热对流上升,冷流体下降”的流动基本特征.最简单的例子就是盛水容器底部加热所观测到的热对流和沸腾现象.因为热力分布的不均匀,造成了流体中的温差,而温差又可引起流体的密度差,于是在重力场中相应地出现了阿基米德浮力,最后驱动热对流的产生.1900年,贝纳对具有自由面-固壁底层的流体薄层进行了热对流实验,观测到各种对流图形.耗散结构理论的提出使得人们对此类问题有了更系统更深入的认识.

本实验通过观察Rayleigh–Bénard对流现象的产生、发展和演化过程,了解耗散结构理论及非线性动力学的基本内容。具体而言,实验中通过控制 \SI{2}{\mm}, \SI{4}{\mm} 薄层去离子水的上下表面温差,观察到了非线性对流从无到有、从稳定有序到失稳湍流的转变过程,并借此图像定性地分析了系统相应结构的特征。
\\\\
{\bf{关键词}:}\ 热对流,非线性斑图,耗散

%%%%%%%%%%%%%%%%%%%%%%%%%%%%%%%%%%%%%%%% CONTENT %%%%%%%%%%%%%%%%%%%%%%%%%%%%%%%%%%%%%%%%
\newpage
\begin{center}
\tableofcontents\label{c}
\end{center}
\newpage

%%%%%%%%%%%%%%%%%%%%%%%%%%%%%%%%%%%%%%%% Introduction %%%%%%%%%%%%%%%%%%%%%%%%%%%%%%%%%%%%%%%%
\section{引言} \label{overview}%------------------------------
自然界中存在丰富的自组织现象,即由局域的相互作用、从无序初态出发,自发地产生出更大尺度上的有序结构。常见的例子有热对流结构(\textit{如大气环流}),结晶过程(\textit{如冰晶形成}),化学振荡(\textit{如碘钟}),生命过程(\textit{大至进化,小至蛋白质的合成与折叠})等等。从对称性的角度上看,自组织现象往往与所谓\mbox{\textit{自发对称性破缺}}密切相关;这一观点在理论物理当中十分关键,它描述了统一理论到各种相互作用的过渡,如电弱(electroweak)理论到电、弱相互作用的分离过程。
	
由于考虑的往往是开放系统,上述现象与热力学第二定律暗示的无序演化恰好相反。虽说上述自组织现象往往可以通过具体的动力学加以分析,人们更希望以一种统一的语言对上述规律加以描述。正是基于此,Ilya Prigogine于上世纪60年代发展出了耗散结构(dissipative structure)理论, 这一工作于1977年获得Nobel化学奖。

据教材, 耗散结构理论表明,当开放系统偏离平衡达一定程度时,系统将会出现分岔行为,此后系统将离开原先无序的热力学分支,发生突变进入到一个全新的稳定有序状态,此即耗散结构;若将系统拉开到离平衡态更远的地方,系统可能出现更多新的稳定有序状态。理论指出,系统从无序状态过渡到耗散结构有两个必要条件:
\begin{enumerate}[1.]
\item \textbf{系统必须开放},与外界进行物质或能量交换;
\item \textbf{系统必须远离平衡},突破近平衡的线性动力学而采用非线性动力学。
\end{enumerate}

动力学上,在系统的非线性效应很小时,系统内部出现的随机涨落会衰减,但当系统的非线性效应不可忽略时,系统内部的产生的随机涨落会会逐渐变大,自组织后形成稳定的耗散结构。
	
本实验观察的Rayleigh–Bénard对流正是耗散结构的一个具体实例。对流由薄液层的上下温差驱动,在突破静态的稳定临界条件时出现;此时通过阴影法可观察到有序的斑图。当温差过大时,耗散结构崩溃而被湍流所取代。本实验通过观察对流版图的形成与演化,以初步了解非线性动力学的基本特征,同时检验耗散结构的上述特性。

%%%%%%%%%%%%%%%%%%%%%%%%%%%%%%%%%%%%%%%% Theory %%%%%%%%%%%%%%%%%%%%%%%%%%%%%%%%%%%%%%%%
\newpage
\section{理论} \label{theory}%------------------------------
研究在上下相隔$d$的两个平面之间所夹住的一薄层液体中的热对流现象.上下两边界为水平,温度分别维持
为$T$和$T+\Delta T$($\Delta T > 0$).这一热对流系统满足Boussinesq条件,相应的热对
流基本方程组和边界条件如下

\begin{equation}
    \begin{cases}
        \frac{\partial \vec{V}}{\partial t} + \vec{V} \cdot \nabla\vec{V} =
        g\alpha T\hat{z} - \frac{1}{\rho}\nabla p + \gamma\nabla^2\vec{V} \\
        \nabla\cdot\vec{V} = 0 \\
        \frac{\partial \theta}{\partial t} + \vec{u}\cdot\nabla\theta =
        \kappa\nabla^2 T
    \end{cases}
\end{equation}

\begin{equation}
    \begin{cases}
        T(z = -d/2) = T + \Delta T \\
        T(z = d/2) = T \\
        \vec{V} = 0, z = \pm d/2
    \end{cases}
\end{equation}

对于这一定解问题可以做线性稳定性分析,通过对定态解施加围绕,讨论微扰的线性发展.

理论推导给出的结果指出,参数$R$瑞利系数
\begin{equation}
    R = \frac{g\alpha d^3\Delta T}{\kappa\gamma}
\end{equation}
是决定系统非线性特性的重要参量.数值计算结果则给出,$R$存在临界值
\begin{equation}
    R_c = 1707.76, \alpha_c = 3.117
\end{equation}
即是说,当系统的条件可以满足$R<R_c$时, 系统出现的随机扰动噪声会随时间演化消失,从
而不会出现非线性对流现象;而系统的条件满足$R>R_c$时,系统出现的随机扰动的非线性效
应将不会能够忽略,从而将会出现非线性对流现象与非线性斑图.
%%%%%%%%%%%%%%%%%%%%%%%%%%%%%%%%%%%%%%%% Experiment %%%%%%%%%%%%%%%%%%%%%%%%%%%%%%%%%%%%%%%%
\newpage
\section{实验} \label{experiment}%------------------------------
\subsection{实验仪器}\label{sub:instruments}
\begin{itemize}
\item{\textbf{CCD}}
\item{\textbf{电脑}}
\item{\textbf{接收屏}}
\item{\textbf{半反半透镜}}
\item{\textbf{凸透镜}}
\item{\textbf{扩束镜}}
\item{\textbf{激光}}
\item{\textbf{铜盘}}
\item{\textbf{硅胶加热片}}
\item{\textbf{控温计}}
\item{\textbf{电流表}}
\end{itemize}

%------------------------------------------------------------
\subsection{简要实验步骤}\label{sub:ExperimentalSteps}
分为以下几个步骤:\\\\
\circled{1}调整光路,使得接收屏上出现圆形红色图案.并开启电脑,进入winXP系统,打开桌面上软件,观察接收屏上的图像\\\\
\circled{2}接通水泵的开关,确定上层降温水层流动正常,观察出水口是否有水流出\\\\
\circled{3}确定待观测对流水层(置于铜盘和蓝宝石片之间的用黑色圆环,厚为2mm,固定的水层)已加好\\\\
\circled{4}开启温控仪,温控仪A的数字显示对流水层上表面的当前温度,即室温.温控仪B的数字显示为对流水层下表面温度,未加热时应为室温\\\\
\circled{5}开启制冷机、控制水箱内水温、较室温略低即可,此温度一致控制不变,保持恒定\\\\
\circled{6}改变硅胶片加热电流,从而改变温度差,观察并记录温差和相对应的稳定斑图\\\\
\circled{7}重置4mm圆环的厚度水层,观察版图随温差增加的状态演化\\\\

%%%%%%%%%%%%%%%%%%%%%%%%%%%%%%%%%%%%%%%% Results & Discussions %%%%%%%%%%%%%%%%%%%%%%%%%%%%%%%%%%%%%%%%
\newpage
\section{结果及讨论}
%------------------------------------------------------------
\subsection{观察2mm圆环的厚度水层上斑图随温差增加的状态演化}\label{sub:1}
加水后开始时斑图如下\textbf{图\ref{result:fig1}}所示结果 
\begin{figure}[H]
 \centering
 \caption{I=0.001A,\ $\Delta T=\SI{0.8}{\celsius}$}
 \includegraphics[height=8cm, width=10cm]{images/25.3_26.1_0.001_3.19.bmp}
 \label{result:fig1}
\end{figure}
接着演化到如下\textbf{图\ref{result:fig2}}所示结果 
\begin{figure}[H]
 \centering
 \caption{I=0.199A,\ $\Delta T=\SI{1.2}{\celsius}$}
 \includegraphics[height=8cm, width=10cm]{images/25.5_26.7_0.199_3.30.bmp}
 \label{result:fig2}
\end{figure}
以上两张图可以看出还没有任何斑图出现
\newpage
接着演化到如下\textbf{图\ref{result:fig3}}所示结果 
\begin{figure}[H]
 \centering
 \caption{I=0.698A,\ $\Delta T=\SI{4.1}{\celsius}$}
 \includegraphics[height=8cm, width=10cm]{images/27.3_31.4_0.698_3.40.bmp}
 \label{result:fig3}
\end{figure}
接着演化到如下\textbf{图\ref{result:fig4}}所示结果
\begin{figure}[H]
 \centering
 \caption{I=0.751A,\ $\Delta T=\SI{5.3}{\celsius}$}
 \includegraphics[height=8cm, width=10cm]{images/28.1_33.4_0.751_3.50.bmp}
 \label{result:fig4}
\end{figure}
以上两张图可以看出仍然没有任何斑图出现
\newpage
接着演化到如下\textbf{图\ref{result:fig5}}所示结果
\begin{figure}[H]
 \centering
 \caption{I=0.800A,\ $\Delta T=\SI{6.1}{\celsius}$}
 \includegraphics[height=8cm, width=10cm]{images/28.9_35.0_0.800_4.00.bmp}
 \label{result:fig5}
\end{figure}
接着演化到如下\textbf{图\ref{result:fig6}}所示结果
\begin{figure}[H]
 \centering
 \caption{I=0.846A,\ $\Delta T=\SI{6.7}{\celsius}$}
 \includegraphics[height=8cm, width=10cm]{images/31.0_37.7_0.846_4.36.bmp}
 \label{result:fig6}
\end{figure}
以上两张图可以看到已经接近斑图出现的临界区域
\newpage
接着演化到如下\textbf{图\ref{result:fig7}}所示结果
\begin{figure}[H]
 \centering
 \caption{I=0.903A,\ $\Delta T=\SI{7.2}{\celsius}$}
 \includegraphics[height=8cm, width=10cm]{images/31.8_39.0_0.903_4.47.bmp}
 \label{result:fig7}
\end{figure}
接着演化到如下\textbf{图\ref{result:fig8}}所示结果
\begin{figure}[H]
 \centering
 \caption{I=1.196A,\ $\Delta T=\SI{9.9}{\celsius}$}
 \includegraphics[height=8cm, width=10cm]{images/34.6_44.5_1.196_5.00.bmp}
 \label{result:fig8}
\end{figure}
以上两张图可以看到较清晰的斑图图样出现
\newpage
接着演化到如下\textbf{图\ref{result:fig9}}所示结果
\begin{figure}[H]
 \centering
 \caption{I=1.490A,\ $\Delta T=\SI{13.3}{\celsius}$}
 \includegraphics[height=8cm, width=10cm]{images/38.4_51.7_1.490_5.11.bmp}
 \label{result:fig9}
\end{figure}
这张图可以看到非常清晰的斑图图样出现,此时的上下表面温度差已经超过了斑图出现的阈值
%------------------------------------------------------------
\subsection{观察4mm圆环的厚度水层上斑图随温差增加的状态演化}\label{sub:2}
加水后开始时斑图如下\textbf{图\ref{result:fig10}}所示结果 
\begin{figure}[H]
 \centering
 \caption{I=0.001A,\ $\Delta T=\SI{0.7}{\celsius}$}
 \includegraphics[height=8cm, width=10cm]{images/30.9_31.6_0.001_5.43.bmp}
 \label{result:fig10}
\end{figure}
这张图可以看出还没有任何斑图出现
\newpage
接着演化到如下\textbf{图\ref{result:fig11}}所示结果
\begin{figure}[H]
 \centering
 \caption{I=0.203A,\ $\Delta T=\SI{0.7}{\celsius}$}
 \includegraphics[height=8cm, width=10cm]{images/30.7_31.4_0.203_5.51.bmp}
 \label{result:fig11}
\end{figure}
接着演化到如下\textbf{图\ref{result:fig12}}所示结果
\begin{figure}[H]
 \centering
 \caption{I=0.599A,\ $\Delta T=\SI{2.2}{\celsius}$}
 \includegraphics[height=8cm, width=10cm]{images/32.0_34.2_0.599_6.07.bmp}
 \label{result:fig12}
\end{figure}
以上两张图可以看出仍然没有任何斑图出现
\newpage
接着演化到如下\textbf{图\ref{result:fig13}}所示结果
\begin{figure}[H]
 \centering
 \caption{I=0.793A,\ $\Delta T=\SI{3.5}{\celsius}$}
 \includegraphics[height=8cm, width=10cm]{images/33.7_37.2_0.793_6.18.bmp}
 \label{result:fig13}
\end{figure}
接着演化到如下\textbf{图\ref{result:fig14}}所示结果
\begin{figure}[H]
 \centering
 \caption{I=1.092A,\ $\Delta T=\SI{5.4}{\celsius}$}
 \includegraphics[height=8cm, width=10cm]{images/36.9_42.3_1.092_6.27.bmp}
 \label{result:fig14}
\end{figure}
以上两张图可以看到已经接近斑图出现的临界区域,并且有初步清晰的斑图出现
\newpage
接着演化到如下\textbf{图\ref{result:fig15}}所示结果
\begin{figure}[H]
 \centering
 \caption{I=1.486A,\ $\Delta T=\SI{9.2}{\celsius}$}
 \includegraphics[height=8cm, width=10cm]{images/43.0_52.2_1.486_6.40.bmp}
 \label{result:fig15}
\end{figure}
接着演化到如下\textbf{图\ref{result:fig16}}所示结果
\begin{figure}[H]
 \centering
 \caption{I=1.787A,\ $\Delta T=\SI{11.8}{\celsius}$}
 \includegraphics[height=8cm, width=10cm]{images/48.0_59.8_1.787_6.47.bmp}
 \label{result:fig16}
\end{figure}
以上两张图可以看到非常清晰的斑图图样出现,此时的上下表面温度差已经超过了斑图出现的阈值
%add more subsections for other block
%%%%%%%%%%%%%%%%%%%%%%%%%%%%%%%%%%%%%%%% Conclusion %%%%%%%%%%%%%%%%%%%%%%%%%%%%%%%%%%%%%%%%
\newpage
\section{结论}\label{conclusions}
本实验通过考察耗散结构的经典实例:Rayleigh–Bénard对流斑图;实验中比较了$d = \SI{2}{\mm},\SI{4}{\mm}$薄层去离子水的上下表面温差,观察分析了斑图形成至崩溃的过程,认识并检验了耗散结构的基本特征,初步了解了非线性动力学的分析手段。

%%%%%%%%%%%%%%%%%%%%%%%%%%%%%%%%%%%%%%%% Questions %%%%%%%%%%%%%%%%%%%%%%%%%%%%%%%%%%%%%%%%
\begin{comment}
\section{实验报告思考题}\label{questions}
\subsection{在$a=23.0mm$、$b=10.0mm$的矩形波导管中能不能传播$\lambda=2cm$、$3cm$和$5cm$的微波?各能传播哪些波型?}\label{sub:question1}
答:根据
\begin{equation}
    \lambda_c=\frac{2}{\sqrt{(m/a)^2+(n/b)^2}}
\end{equation}
我们可以算出可传播的最大波长为$\lambda_{max}=18.3mm$,显然不能传播$\lambda=2cm$、$3cm$和$5cm$的微波,可传输波长在$\lambda_{max}=18.3mm$以下,满足$\lambda_c=\frac{2}{\sqrt{(m/a)^2+(n/b)^2}}$的波长的波型\\

\end{comment}

%%%%%%%%%%%%%%%%%%%%%%%%%%%%%%%%%%%%%%%% Acknowledgements %%%%%%%%%%%%%%%%%%%%%%%%%%%%%%%%%%%%%%%%

\section{致谢}\label{acknowledgments}
感谢老师在实验中的的悉心指导.

\begin{comment}
%%%%%%%%%%%%%%%%%%%%%%%%%%%%%%%%%%%%%%%% Appendix %%%%%%%%%%%%%%%%%%%%%%%%%%%%%%%%%%%%%%%%
\appendix
\section{代码}\label{sub:app.code}
请在附录\ref{sub:app.code}中添加代码。请使用如下Scala的语法高亮描述方法。
\begin{scala}
class TopIO extends Bundle() {
	val boot = Input(Bool()) 
// imem and dmem interface for Tests
	val test_im_wr		= Input(Bool())
	val test_im_rd 		= Input(Bool())
	val test_im_addr 	= Input(UInt(32.W))
	val test_im_in 		= Input(UInt(32.W))
	val test_im_out 	= Output(UInt(32.W))

	val test_dm_wr		= Input(Bool())
	val test_dm_rd 		= Input(Bool())
	val test_dm_addr 	= Input(UInt(32.W))
	val test_dm_in 		= Input(UInt(32.W))
	val test_dm_out 	= Output(UInt(32.W))

	val valid			= Output(Bool())
}
class Top extends Module() {
	val io 		= IO(new TopIO())//in chisel3, io must be wrapped in IO(...) 
	//...
	when (io.boot & io.test_im_wr){
		imm(io.test_im_addr) := io.test_im_in
		} .elsewhen (io.boot & io.test_dm_wr){
		// please finish it
		} //...
}
\end{scala}
\newpage
\end{comment}
%%%%%%%%%%%%%%%%%%%%%%%%%%%%%%%%%%%%%%%% REFERENCE %%%%%%%%%%%%%%%%%%%%%%%%%%%%%%%%%%%%%%%%
\bibliographystyle{unsrt}
\bibliography{Ref}
\nocite{*}

\end{document}

